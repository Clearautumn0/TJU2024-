\chapter{项目开发过程描述}

为了方便项目的开发进度掌握,我们将整个项目开发周期的三周划为五个阶段,这里的阶段指每次集中向老师汇报的中间时间段,集中汇报的时间为每周的周三以及周日晚,一阶段时常为三至四天,具体实现安排为:
\begin{itemize}
    \item 第一阶段:(8.27-8.30)
    \item 第二阶段(8.31-9.4)
    \item 第三阶段(9.5-9.8)
    \item 第四阶段(9.9-9.11)
    \item 第五阶段(9.12)
\end{itemize}
其中第一阶段为学习阶段,第二三阶段为解决已知问题阶段,第四阶段为新内容开发以及文档整理阶段,第五阶段为部署阶段
\section{第一阶段:(8.27-8.30)}
本阶段属于项目初期阶段,主要内容是学习智慧树上提供的相关内容,成员学习前后端相关知识,搭建项目开发环境,初步起草文档内容,完成项目的分工安排以及实践安排.
\begin{enumerate}
    \item 建立gittee仓库
    \item 组员分工
    \item 下载老师提供的饿了么V1.0工程文件
    \item 组员在各自电脑上完成环境的安装,试运行饿了么V1.0的前后端项目
    \item 开始撰写软件需求规划说明书(SRS)初稿
    \item 整理出符合RESTful分割的前后端API接口信息 \href{https://www.yuque.com/clearautumn/gse398/ecqwgdmq8ylnntda#F75L}{约束文档}
    \item 一次组会
\end{enumerate}
\subsection{第一次组会(8.27)记录}
\begin{itemize}
    \item 对成员分工进行详细的安排
    \item 建立gittee仓库
    \item 组员分方向学习智慧树内容
    \item 组员各自配置开发环境
    \item 启动SRS的初稿撰写
\end{itemize}

\subsection{第二次组会(8.29)记录}
\begin{itemize}
    \item 成员分享智慧树学习进度
    \item 成员展示本地环境安装,王俊哲提供安装指导
    \item 王俊哲介绍RESTful风格API的相关内容,规定后续接口实现方式
    \item 展示本阶段工作成果
\end{itemize}
\section{第二阶段(8.31-9.4)}
第二阶段是项目的启动阶段,在这一阶段,成员将解决饿了吧V1.0中遗留问题的较为简单的部分,熟悉开发环境,熟悉合作开发的模式,熟悉前后端代码,为后续的开发练手.

    \begin{enumerate}
 \item  完善SRS的内容(整理RESTful风格化后的API接口信息).
 \item  建立github仓库 ,设置.git/config配置文件,同时推送到gittee与github仓库.
 \item  负责前端的成员继续学习关于前端编程的内容,了解vue2.0与vue3.0的区别,为后续将项目前端技术从VUE2升级到VUE3作基础.
 \item  对数据库结构进行优化并整理出数据库结构一览文档,便于API接口RESTful风格化以及后续的开发.
 \item  初步启动"饿了吧V2.0"项目.
 \item  实现对"饿了吧V1.0"中前后端API进行符合RESTful风格的修改.
 \item  使用latex对SRS等文档的内容进行规范化排版.
 \item  本阶段预期解决"饿了吧V2.0"项目中的如下问题:
    \begin{enumerate}
        \item   页面无法返回上一步。首页之后的页面缺少返回键,用户只能一路点到底才能退出。
        \item   前端订单页面的小数显示上存在 bug,可能会出现小数点后很多位数字,没有使用 toFixed() 函数进行更改限制.
        \item   在对订单中涉及的商品名称、价格等信息进行修改时,会引起历史订单信息的修改,应在存储历史信息时进行拷贝存储。修改送货地址也有类似问题。
        \item   前端版本代码均存在商家列表/⻝品列表不能拉到底,最后⼀个信息会被底部菜单栏遮挡的问题.
        \item   历史未支付订单应该有支付入口.
        \item   手机号的验证不严谨,只有对是否为空的验证,没有对手机号的位数,开头数字,以及运行商信息进行验证.(后端二次)
        \item  购物车不能展开,无法查看自己已经买的东西的单价和数量。
        \item  后端向前端返回的对象为数据库中的该对象的全部信息,例如返回给前端的⽤户实体包含该⽤户密码,这是⼀个不安全的设计,直接将⼀些不该暴露的信息暴露给前端,存在巨⼤的信息安全隐患,应当把向前端返回的信息单独封装进⼀个类,只返回必要的信息。
        \item  后端实现对用户数据的修改与删除流程,实现商家的注册、修改、删除流程,实现商品的添加、修改、删除流程。
        \item  10)初步添加"我的"以及"个人信息""页面.
    \end{enumerate}
    \end{enumerate}
\subsection{第三次组会(9.1)记录}
\begin{itemize}
    \item 组织开发分工.
    \item 后端介绍RESTful风格化之后的后端代码结构以及后续接口设计规范.
    \item 前端展示返回功能的效果,效果不佳,勒令整改.
    \item 讨论TDD开发模式,指定初步的测试代码方案:对于后端代码进行测试.
\end{itemize}
\subsection{第四次组会(9.4)记录}
\begin{itemize}
    \item 成员展示各自任务的实现效果
    \item 对下阶段任务做出安排
\end{itemize}


\section{第三阶段(9.5-9.8)}

本阶段是项目的基本完成阶段,将解决教师提供的文档中提出的所有问题.
  \begin{enumerate}
\item 本阶段预计实现饿了吧V1.0到饿了吧V2.0的全部功能.
\item 本阶段实现VUE2到VUE3的升级.
\item 检查项目本身存在的bug,并进行一一修复,包括:
\begin{itemize}
  \item 注册用户时,输入手机号一栏失去焦点时抛出异常的bug.
  \item 用户注册时,错误格式的手机号在失去焦点并点击注册之后会弹窗两次提醒.
    \end{itemize}
\item 使用latex对已有文档的规范化排版.
\item  本阶段预期解决如下问题:
   \begin{enumerate}
       \item 在后端添加对手机号的二次验证( 注意对于验证失败的处理方式).
       \item 前端添加"我的"页面对用户信息的拉取(包括头像和用户名等信息).
       \item 用户隐私信息如:电话号码、登录密码等,通过接口对参数进行明文传输,这使得用户的隐私信息被直接暴露,造成极大的安全隐患。
       \begin{itemize}
       \item  这里使用HTTP POST传输, 避免在 URL 中传输敏感数据.
       \item 使用rsa密钥,在前端使用后端发送的公钥进行加密,在后端使用私钥解密.
          \end{itemize}
       \item ⽤户注册时,存⼊的密码未经任何加密处理,这⾥应该将密码单向加密后再存⼊数据库。
       \item 后端对前端提交的数据和身份并不检验,这就导致可能有伪造的前端提交恶意数据,可能修改他人送餐地址,或者修改订单金额等。
       \item 在header添加token实现身份认证以及信息安全传输.
       \item 商品不是无限的,应该有数量,订购时应加互斥锁(这里通过对问题的分析发现,生活中外卖软件实际上并未对商品的数量进行限制,而是商家手动设置商品是否告罄,于是此处问题我们转换为商品的订购要事先查询该商品是否告罄).
       \begin{itemize}
       \item 修改数据库结构,在food表中加入新字段soldOut.
       \item 对dao层的sql语句进行修改.
       \end{itemize}
       \item 考虑到在实际应用场景中可能存在网络异常情况,例如用户进行订单支付时接口超时,重新调用接口会导致重复扣款,严重影响软件质量及用户评价。
       \item  后端代码订单业务逻辑缺少对订单和购物⻋是否为空的验证.
   \end{enumerate}
 \end{enumerate}

\subsection{第五次组会(9.5)记录}
 \begin{enumerate}
     \item 后端介绍token的开发思路以及在后端时实现方式.
     \item 安排阶段任务,本阶段将完成所有内容,下一阶段着手新内容的开发.
     \item 前端成员指出前端目前存在得到页面bug
 \end{enumerate}

\subsection{第六次组会(9.7)记录}
\begin{itemize}
    \item 成员介绍各自任务完成情况
    \item 成员共同对数据库结构做出修改
    \item 覃邱维介绍Rsa密钥密码加密原理
    \item 成员商议商品售罄的处理办法
\end{itemize}

\subsection{第七次组会(9.8)记录}
\begin{itemize}
    \item 成员商议下次组会的汇报人选
    \item 成员展示各自任务完成情况
\end{itemize}




\section{第四阶段(9.9-9.11)}
本阶段是创新阶段,项目经理根据自己的理解对项目进行优化,力求在有限的实践内实现尽可能多的内容,本阶段临近项目验收,也是准备答辩ppt以及收拢文档内容的总要阶段.
 \begin{enumerate}
     \item 前端对弹窗信息方式进行修改.
     \item 制作支付成功页面并添加路由,实现已支付订单信息的显示.
     \item 我的界面从后端拉取头像的修改.
     \item 后端修改手机号验证的返回结果多样性.
     \item 后端对数据库结构进行修改:用户添加用户权限字段:(0:管理员,1:普通用户,2:商家用户),并对后端部分内容做出修改.
     \item 准备开始做商家上架以及下架商品的功能(前端+后端).
     \item 前端添加逻辑:当订购商品数量达到50报请求数据失败异常,跳转人机验证.
     \item 前端完成搜索框的实现.
     \item 整理更新文档内容.
     \item 着手准备结课报告以及答辩ppt.
     \item 前端设计订单超时取消逻辑,设计手动取消订单的接口.
     \item 后端添加更改订单状态的接口.
 \end{enumerate}

 \subsection{第八次组会(9.10)记录}
\begin{itemize}
    \item 项目经理介绍新内容的开发思路
    \item 成员针对新内容开发提供设计思路
\end{itemize}
 \subsection{第9次组会(9.11)记录}
\begin{itemize}
    \item 集中调试修改bug
    \item 安排下一阶段任务
\end{itemize}
\section{第五阶段(9.12)}
本阶段是最后的收尾阶段,在这个阶段要完成项目的最终测试以及项目的部署,整理出文档终稿,制作项目汇报ppt.
 \subsection{第10次组会(9.12)记录}
\begin{itemize}
    \item 成员讨论汇报内容
    \item 展示部分功能实现原理,为汇报展示做准备
    \item 成员集中校验文档内容
\end{itemize}