\chapter{成员分工合作说明}

小组共四名成员,\textbf{覃邱维,王俊哲,聂哲浩,李亮克.}

项目初期,一名成员担任项目经理,统筹管理成员工作,辅助撰写小组文档,安排阶段任务以及每日安排任务,负责项目汇报.一名成员负责后端代码工作,同时担任技术指导,一名成员负责前端代码工作,一名同学负责主要文档的撰写.

到了项目中期,我们小组发现前端代码的工作比较多,于是安排两位同学负责前端,两位同学负责后端内容,其中项目经理辅助后端同学进行后端开发已经完成文档内容的撰写.

到了项目后期,负责后端的两位同学开始兼顾前端内容,帮助前端同学进行前端内容的设计以及代码实现.

\begin{itemize}
    \item 由覃邱维担任项目经理,负责安排组员每日任务以及阶段任务,组织开展组会,负责文档的撰写大部分内容,完成latex排版,在中期参与后端与前端内容的撰写,负责项目汇报.是项目的组织者.
    \item 王俊哲负责后端代码以及技术指导,后端的大部分逻辑与内容的开发,初期整理出符合RESTful风格接口的约束文档,参与到SRS的撰写,指导组员的环境安装.在中期参与前端的逻辑开发,到了后期是前端开发的逻辑部分主要实现人.
    \item 聂哲浩在初期参与到SRS的撰写,在中后期参与VUE升级工作以及前端UI的设计与开发。页面设计:个人主页,个人资料页面,店铺管理页面,上架商品页面,下架商品页面,搜索页面。另外采用pytest+Selenium实现前端页面功能测试。优化了前端的多个页面风格,使其变得易于操作和易于管理.
    \item 李亮克负责前端的设计与开发以及管理,完成了课程任务要求中前端任务的绝大多数内容,如页面中“返回”按键的添加、订单小数位数显示格式的处理,购物车功能实现,添加新的页面并实现对用户个人信息如手机号用户名的拉取功能等。同时完成了项目自带的问题。除此之外,参与了前端新增内容的实现,如用户取消订单,支付订单、以及上架商品功能。同时和聂哲浩合作将项目从vue2升级到了vue3、.
\end{itemize}