\chapter{问题,对策与反思}
\section{前端部分}
\begin{enumerate}
    \item  返回按钮:\par
    返回按钮我们采用的是小组件的方式,类似于源代码中的Footer.vue,但是随着开发的一步步发展,返回按钮的适用性得到了非常多的考验,为此我们对页面功能做了很多的调整,诸如设置跳转判断,和禁用返回键。
    \item  路由管理:\par
    管理elm项目的过程中对于页面各个功能的协调造成了路由混乱的情况,对此我们队路由的先决条件做了很多的修改,诸如判断用户当前的状态信息以及当前页面的跳转需求。
    \item  接口调用:\par
    随着项目功能的进一步增加,我们队API接口的调用次数以及调用范围逐渐扩大,这就伴随着调用方式有误或者前端业务需求满足的问题,对于这些问题,后端做了很多的管理和修改工作。
    \item  页面布局:\par
    前端为了使页面更加整洁优美,需要对页面的布局不断地做出调整,调整的过程异常层出不穷,细节方面的调整做了不少,后期我们采用了Element工具包来优化页面,使页面变得更加优美和易于操作。
    \item  Vue版本升级:\par
    项目文件的起始版本为Vue2,我们需要按照项目计划书要求将项目升级为Vue3,以适应新的功能。升级过程终于到了很多的问题,诸如运行时报错访问到未定义的变量,为此经过了非常久的调试找出了原因,即vue2和3对于v-if和v-for的渲染有先后顺序的区分;运行时一些script钩子函数的渲染时机与Vue2时不同导致页面无法正常显示,。同时一些依赖包的安装和应用过成功出现了很多的报错,诸如安装位置冲突造成新的包不能正常安装。同时兼容组件的使用报错也是层出不穷,在多次的调试和修改后升级为Vue3。
\end{enumerate}
\section{后端部分}
\subsection{历史订单信息的存储}
在本次项目中对于历史信息的存储是一个难点,对于历史信息存储的实现方法是一个难题,小组讨论中,建议在不增添新表的情况下实现这个功能,最终决定使用"假删除"和"假更新"的办法解决,具体实现见项目特色介绍\ref{con:6.03}

